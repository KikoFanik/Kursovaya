\documentclass[14pt, oneside]{altsu-report}

\worktype{Отчёт по практике на тему:}
\title{Лабиринт}
\author{В.\,А.~Суслова}
\groupnumber{5.205-1}
\GradebookNumber{1337}
\supervisor{И.\,А.~Шмаков}
\supervisordegree{}
\ministry{Министерство науки и высшего образования}
\country{Российской Федерации}
\fulluniversityname{ФГБОУ ВО Алтайский государственный университет}
\institute{Институт цифровых технологий, электроники и физики}
\department{Кафедра вычислительной техники и электроники}
\departmentchief{В.\,В.~Пашнев}
\departmentchiefdegree{к.ф.-м.н., доцент}
\shortdepartment{ВТиЭ}
\abstractRU{Проект по дисциплине "Проекто-технологическая практика" на тему "Лабиринт".
\par Работа изложена на *** страницах машинописного текста.
\par Краткое содержание работы. Данная работа посвящена разработке, тестировке игры "Лабиринт".}
\abstractEN{A project on the discipline "Design and technological practice" on the theme "Labyrinth".
\par The work is presented on *** pages of typewritten text.
\par Summary of the work. This work is devoted to the development and testing of the game "Labyrinth"}
\keysRU{компьютерное моделирование, cистема управления версиями}
\keysEN{computer simulation, distributed version control}

\date{\the\year}

% Подключение файлов с библиотекой.
\addbibresource{graduate-students.bib}

% Пакет для отладки отступов.
%\usepackage{showframe}

\begin{document}
\maketitle

\setcounter{page}{2}
\makeabstract
\tableofcontents

\chapter*{Введение}
\phantomsection\addcontentsline{toc}{chapter}{ВВЕДЕНИЕ}

\textbf{Актуальностью} темы состоит в том, что в настоящее время разработка игр является развивающейся областью информационных технологий и изучение опыта создания игр может быть полезно для разработки новых проектов.

\textbf{Целью} данного исследования является разработка и создание интересной игры "Лабиринт", освоение технологии программирования.

\par Для достижения цели были поставлены следующие \textbf{задачи:}
\begin{enumerate}
\item Изучить основные принципы и методы создания компьютерных игр;
\item Разработать дизайн и концепцию игры "Лабиринт";
\item Реализовать основные игровые механики;
\item Осуществить программировани игры на языке Python;
\item Разработать интерфейс и графическое оформление;
\item Протестировать игру и исправить возможные ошибки.
\end{enumerate}

\chapter*{Глава 1. }
\phantomsection\addcontentsline{toc}{chapter}{ГЛАВА 1}
% Подключение первой главы (теория):
\include{chapter-1-report-csae.tex}

\chapter*{Глава 2. }
\phantomsection\addcontentsline{toc}{chapter}{ГЛАВА 2}
% Подключение второй главы (практическая часть):
\include{chapter-2-report-csae.tex}

\chapter*{Глава 3. Тестирование игры}
\phantomsection\addcontentsline{toc}{chapter}{ГЛАВА 3. тЕСТИРОВАНИЕ ИГРЫ}
% Подключение третий главы (практическая часть с тестированием:
\include{chapter-3-report-csae.tex}

\chapter*{Заключение}
\phantomsection\addcontentsline{toc}{chapter}{ЗАКЛЮЧЕНИЕ}

\chapter*{Список использованной литературы}

\begin{enumerate}
\item \title = {Буйначев С.К., Боклаг Н. Ю. Основы программирования на языке Python. — Екатеринбург: <<Издательско-полиграфический центр УрФУ>>, 2014. — 91 с.}
\item \title = {Болотский А.В. Математическое программирование и теория игр. — Санкт-Петербург: <<Лань>>, 2022. — 116 с.}
\item \title = {Роберт Нистрем Паттерны программирования игр. — Москва: <<БОМБОРА>>, 2022. — 432 с.}
\item \title = {Эрик Мэтиз Изучаем Python: программирование игр, визуализация данных, веб-приложения. — Санкт-Петербург: <<Питер>>, 2020. — 512 с.}
\item \title = {Мэтт Харрисон Как устроен Python. Гид для разработчиков, программистов и интересующихся. — Санкт-Петербург: <<Питер>>, 2019. — 272 с.}
\end{enumerate}

\newpage
\phantomsection\addcontentsline{toc}{chapter}{СПИСОК ИСПОЛЬЗОВАННОЙ ЛИТЕРАТУРЫ}

\printbibliography[title={Список использованной литературы}]


\appendix
\newpage
\chapter*{\raggedleft\label{appendix1}Приложение}
\phantomsection\addcontentsline{toc}{chapter}{ПРИЛОЖЕНИЕ}
%\section*{\centering\label{code:appendix}Текст программы}

\begin{center}
\label{code:appendix}Текст программы
\end{center}

\begin{code}
\captionof*{listing}{\centering\label{code:pi-example}Пример программы вычисления числа $\pi$ на языке \textit{C} с использованием \textit{MPI} (пример из https://ru.wikipedia
.org/wiki/Message\_Passing\_Interface)}
\vspace{-1cm}\inputminted{C}{src/pi-mpi.c}
\end{code}

\end{document}

